\documentclass[presentation,xcolor={usenames,dvipsnames}]{beamer}

% figures
\newcommand{\home}{.}
\newcommand{\figures}{\home/figures}
\input formatting.tex
\input defs.tex
\usepackage{hyperref}
\hypersetup{colorlinks=true}

% \mode<handout>
% %\mode<presentation>
{
\usetheme{default}
}
\setbeamertemplate{footline}[frame number]

\bibliographystyle{alpha}

\title{ORIE 7191: Optimization for Machine Learning\\[2ex]
       How to Give a Talk}

\date{\textcolor{blue}{\today}}
\author{Professor Udell \\[1ex]
Operations Research and Information Engineering \\
Cornell}

\begin{document}

\begin{frame}
\titlepage
\end{frame}

\section{How to Give a Talk}

\begin{frame}{Tell a story}

\bit
\item what's the problem?
\item preview your result
\item what solutions have others tried?
\item how does your idea work? and how is it different from previous work?
\item gory details / theory / how it works
\item applications / experiments / numerical evidence
\item now what?
\eit

\end{frame}

\begin{frame}{Inspiration for this meta-talk: Heilmeier Catechism}

Heilmeier (DARPA director in 1950's) taught that grant proposals should include:
\bit
\item What are you trying to do? Articulate your objectives using absolutely no jargon.
\item How is it done today, and what are the limits of current practice?
\item What is new in your approach and why do you think it will be successful?
\item Who cares? If you are successful, what difference will it make?
\item What are the risks? How much will it cost? How long will it take?
\item What are the mid-term and final ``exams'' to check for success?
\eit
for a talk, switch all these to the past tense

\end{frame}

\begin{frame}{What's the problem?}

kinds of problem
\bit
\item a real problem (whose?)
\item an ``open'' problem (why does it matter?)
\eit

pro tips:
\bit
\item you'd better ``solve'' the problem by the end of the talk\ldots
\item pictures help grab attention
\eit

\end{frame}

\begin{frame}{Preview your result}

state your contribution
\bit
\item so audience understands what you did \\
(and can decide to pay attention or not)
\item maybe a theorem, maybe a picture, maybe in words
\item might require introducing some key definitions
\eit

pro tips:
\bit
\item leave caveats for later \\ (but mention now if they're major or minor caveats)
\item no one will listen any more if this part is confusing
\eit

\end{frame}

\begin{frame}{What solutions have others tried?}

why compare to related work?
\bit
\item shows audience how hard the problem is
\item helps audience understand what you did \\
(and what you didn't do)
\item keeps your colleagues feeling collegial
\eit

pro tips:
\bit
\item cite all authors by name if $\leq 3$ authors
\item maybe:
\bit
\item use your initial instead of name (eg, Kallus and U 2018)
\item bold your name (eg, Kallus and \textbf{Udell} 2018)
\eit
\eit

\end{frame}

\begin{frame}{How does your idea work?}

\bit
\item this section is usually longest
\item divide into subsections to explain parts of your approach
\item by the end, the audience understands \emph{why} your idea works
\item use \emph{as little} technical machinery as possible
\item provide intuition
\eit

pro tips:
\bit
\item provide quick high level overview \emph{and} details
\item so non-experts \emph{and} experts understand how it works
\item imagine the first year PhD student in the audience
\eit

\end{frame}

\begin{frame}{Gory details}

\bit
\item now you can impress people and lose them
\item make it clear you have technical chops
\item make the experts think you're smart
\item but none of this matters
\item because the audience already understands
\item the important ideas
\eit

pro tips:
\bit
\item omit this part from a public talk
\item possibly also omit from a colloquium
\item definitely include in a job talk
\item probably include for this class, to build endurance
\item you can skip this if your talk is running over
\eit

\end{frame}

\begin{frame}{Experiments}

\bit
\item prove that your ideas work
\item show that they yield a useful solution
\item and that they actually solve the problem
\eit

pro tips:
\bit
\item ask for people to restore their attention (after gory details)
\item make experiment slides self-contained
\bit
\item state experimental settings, label axes and curves clearly, \ldots
\eit
\eit

\end{frame}

\begin{frame}{Know your audience}

imagine your audience
\bit
\item what do they know already? what will they find surprising?
\item often helps to imagine writing the talk for \emph{one particular person} who you know well
\item while giving the talk, look at the person whose face is giving feedback
\item (while listening to a talk, \emph{be} the person giving feedback)
\eit

it's ok to lose (some of) your audience
\bit
\item but you should plan for \emph{who} you'll lose and \emph{when} you'll lose them
\item generally, everyone should understand everything except for the ``gory details''
\item afterwards, tell people when to start paying attention again
\eit

\end{frame}

\begin{frame}{Concluding}

\bit
\item state conclusions
\item state research directions
\item provide references
\item ask for questions
\eit

\end{frame}

\begin{frame}{Style}

technology
\bit
\item \LaTeX / beamer presentations are common in optimization
\item powerpoint / keynote more common in machine learning
\item google slides for collaborative development
\item theorists can make slides that contain only words, equations, and plots
\item systems presentations usually come with fancy pictures and animations
\eit

length
\bit
\item rule of thumb: one minute per slide
\item more if there are lots of pictures
\item less if there's lots of math to explain
\item have sections you can cut easily: gory details, applications
\item know at what time you should arrive at each section
\eit
\end{frame}

\begin{frame}{Style}

words
\bit
\item brevity is the soul of wit
\item don't distract your audience
\item use bullets, not paragraphs
\item beware of line breaks
\item pick a convention and stick to it
\bit \item capitalization, punctuation, phrases vs sentences, etc \eit
\eit

equations
\bit
\item define your terms
\item define as little as possible
\item use words instead of symbols where possible
\eit
\end{frame}

\begin{frame}{Style}

animations
\bit
\item must have semantic meaning
\eit

delivery
\bit
\item speak slowly and clearly
\item require the audience to ask questions
\item show that you're a human \\
(humor, look at audience, ask questions, stop to think)
\item get ready for improvisation!
\eit

\end{frame}

\begin{frame}{More resources}

  \bit
  \item \href{http://www.americanscientist.org/issues/id.877,y.0,no.,content.true,page.1,css.print/issue.aspx}{scientific writing}
  \item \href{http://acmg.seas.harvard.edu/education/presentations/carlton_presentations.pdf}{presentations}
  \eit

\end{frame}

\section{How to write a review}

\begin{frame}{Reviews}

five paragraphs
\bit
\item summarize the paper
\item high level (subjective) evaluation
\item high level criticisms
\item detailed line-by-line comments
\item summary of review
\eit

\end{frame}

\begin{frame}{Summarize the paper}

objective description of the paper
\bit
\item explain the idea of the paper and its contributions
\item use your own words (so the authors can learn something!)
\item be neutral
\eit

\end{frame}

\begin{frame}{High level evaluation}

sujective comments on the paper
\bit
\item is it well written?
\item is it original or incremental?
\item are the claims supported by theory/experiment?
\item who will be interested in these results?
\eit

\end{frame}

\begin{frame}{High level criticisms}

(more) objective criticisms
\bit
\item does the approach make sense?
\item are there major errors or omissions?
\item are the claims supported by evidence?
\item provide a suggested \emph{fix} for every problem you identify
\item be polite!
\eit

\end{frame}

\begin{frame}{Line by line comments}

moving line by line,
\bit
\item when were you confused?
\item were there mistakes in notation, typos, errors, grammatical mistakes?
\item were any figures difficult to understand?
\item missing references?
\eit

\end{frame}

\section{Homework}

\begin{frame}{Homework}

\bit
\item register
\item sign up for slack
\item mark 3 papers you'd like to present by midnight Friday on \\
\href{https://docs.google.com/spreadsheets/d/1eSJn0_ANEXfOsZZrYwHoQ6F00FKBLz4olbKVtOLoE40}{paper selection google doc}
\eit

\end{frame}

\begin{frame}{Scheduling talks}

\bit
\item I'll assign paper presentations this weekend
\item students will upload their assigned paper to CMT
\item students will bid on papers to review
\eit

\end{frame}

\end{document}
